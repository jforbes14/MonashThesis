% This is a LaTeX thesis template for Monash University.
% to be used with Rmarkdown
% This template was produced by Rob Hyndman
% Version: 6 September 2016

\documentclass{monashthesis}

%%%%%%%%%%%%%%%%%%%%%%%%%%%%%%%%%%%%%%%%%%%%%%%%%%%%%%%%%%%%%%%
% Add any LaTeX packages and other preamble here if required
%%%%%%%%%%%%%%%%%%%%%%%%%%%%%%%%%%%%%%%%%%%%%%%%%%%%%%%%%%%%%%%

\author{Joe Bloggs}
\title{Thesis title}
\degrees{B.Sc. (Hons), University of Tangambalanga}
\def\degreetitle{Doctor of Philosophy}
% Add subject and keywords below
\hypersetup{
     %pdfsubject={The Subject},
     %pdfkeywords={Some Keywords},
     pdfauthor={Joe Bloggs},
     pdftitle={Thesis title},
     pdfproducer={Bookdown with LaTeX}
}


\bibliography{thesisrefs}

\begin{document}

\pagenumbering{roman}

\titlepage

{\setstretch{1.2}\sf\tighttoc\doublespacing}

\chapter*{Acknowledgements}\label{acknowledgements}
\addcontentsline{toc}{chapter}{Acknowledgements}

I would like to thank my pet goldfish for \dots

\chapter*{Declaration}\label{declaration}
\addcontentsline{toc}{chapter}{Declaration}

I hereby declare that this thesis contains no material which has been
accepted for the award of any other degree or diploma in any university
or equivalent institution, and that, to the best of my knowledge and
belief, this thesis contains no material previously published or written
by another person, except where due reference is made in the text of the
thesis.

\vspace*{2cm}\par\authorname

\chapter*{Preface}\label{preface}
\addcontentsline{toc}{chapter}{Preface}

The material in Chapter \ref{ch:intro} has been submitted to the journal
\emph{Journal of Impossible Results} for possible publication.

The contribution in Chapter \ref{ch:litreview} of this thesis was
presented in the International Symposium on Nonsense held in Dublin,
Ireland, in July 2015.

\chapter*{Abstract}\label{abstract}
\addcontentsline{toc}{chapter}{Abstract}

This thesis is about \ldots{}

\clearpage\pagenumbering{arabic}\setcounter{page}{0}

\chapter{Introduction}\label{ch:intro}

This research will examine the relationships between socio-demographics
and voting behaviour in Australian federal elections in the 21st
century, in order to answer the questions as to why Australians vote the
way they do, and how this has changed over time. Voting outcomes and
population characteristics are to be gathered at the electorate level,
using spatial analysis to map demographics from censuses falling either
side of an election to that particular election. The study differs from
more common voting studies in three main ways. First, it uses spatial
modelling tools to connect socio-demographic information to an election
from censuses either side of that election. Secondly, it considers
predictive modelling for voter behaviour, and thirdly it combines
information across multiple elections.

In order to match information from each census and election since 2001,
data in the Geographic Information Software (GIS) format will be
analysed and overlayed - the dominant approach in spatial studies.
Methodologies for intersecting and interpreting GIS objects are well
documented, and have been used in previous analysis of Australian voting
behaviour, along with other studies in other fields including; strategic
planning (Valcik, 2012), healthcare (Ye et al. 2017) and geosciences.

\chapter{Literature Review}\label{ch:litreview}

Existing literature on socio-spatial analysis of Australian elections
are limited to examining a single federal election, with
socio-demographic information pulled from the nearest census. This
approach by used by Stimson et al. (2005), and later this approach was
repeated and adapted to an online e-research platform by Liao et al.
(2009). Both of these studies use GIS to overlay the nearest census to
the single federal election of interest, and use information
disaggregated to polling booth locations, which is a finer level of
disaggregation than the areas this study intends to explore - being
Commonwealth Electoral Divisions (CEDs), also known as electorates.

An area that is blatantly vacant in Australian literature is the
examination of these socio-political relationships over time. It appears
that no study has either attempted to create a collection of
socio-demographics for multiple elections, nor has any study overlayed
information to more than one census to a single Australian federal
election - a problem that naturally arises due to the frequency of the
two events.

The previous single-election studies have analysed socio-political
relationships confined to a single electoral division (Forrest 1982), or
a particular political party (Davis and Stimson 1998). Stimson \& Shyy
(2012) expanded on this by examining how key demographic and
socio-economic characteristics affect voter support for political
parties. This was done using the GIS data from the previously mentioned
online facility, and used univariate visualisations, linear regressions,
summary statistics and discriminant analysis to model relationships
between population variables and votes for a particular party.
Discriminant analysis was also used by Stimson et al. (2001) because the
study aimed to distinguish between political parties in their voter
support, rather than predict how areas would vote. Both descriptive and
predictive analytics will be undertaken in this research, as it aims to
only uncover patterns in voters for each party and predict how
electorates would vote with a given set of characteristics.

The methods for descriptive and predictive modelling are outlined in the
methodology section of this submission.

\chapter{Data}\label{ch:Data}

The two main data sources for this research are the Australian Census of
Population and Housing from the Australian Bureau of Statistics (ABS),
and published election results from the Australian Electoral Commission
(AEC).

The Census of Population and Housing collects data on the key
characteristics of every Australian and is conducted every five years.
There have been four censuses in the 21st century, being that in 2001,
2006, 2011 and 2016. All of these are used in this study to provide
socio-demographic information at electorate level. Federal elections
typically occur every three years, and the those of interest will be
from 2001, 2004, 2007, 2010, 2013 and 2016. All information from these
sources is publically available, so this project will be reproducible -
proving a resource for future research.

\section{Commonwealth Electorate
Boundaries}\label{commonwealth-electorate-boundaries}

\textcite{AEC-Overview} For the House of Representatives each State and
Territory is divided into electoral divisions, totalling 150 electorates
across Australia. The number of these divisions is determined by
population. To ensure equal representation, the boundaries of these
divisions have to be redrawn or redistributed periodically.
Redistributions (changing of boundaries) typically only affect a handful
of electorates, with most remaining the same as previously defined.
Redistribution dates can be found in the appendix
\textcite{AEC-DistDates}.

\textcite{ABS-CED} The Commonwealth Electoral Divisions (CED) are an ABS
approimation of the AEC electoral division boundaries. CEDs may change
as the AEC revises boundaries, and the CED update will occur in the
month of July following the AEC changes.

The boundaries may be redistributed between elections, but this has no
effect on the House of Representatives until the next election.

\section{Census}\label{census}

Write a summary of the data taken from each census

\subsubsection{Non-response in Census}\label{non-response-in-census}

Like in any survey, non-response bias is a source of potential problems.
The ABS releases statements with each Census on its data quality, and
for the years considered, this study assumes reliability of the
published data.

It is worthwhile noting that total item non-response increased in 2016,
compared with 2011, for non-imputed items but remains reasonably close
to levels achieved in 2006 \textcite{ABS-CQ16}. The main contributor to
item non-response is people who do not respond to the Census at all.
Amongst those who responded, non-response rates have steady dropped
since 2006. The ABS has imputed key variables (age, sex, martial status
and usual residence) for non-response, although is not clear whether
this has been done in 2001.

Non imputed items are treated as ``not stated'' or ``not applicable'',
dependent on the imputed age of the person.

The assumption of reliability means that no adjustments or imputations
will be explicitly made in this study to the values derived from each
Census. However, the frequency of ``not stated'' responses will be
recorded for particular questions, and will be included with other
Census-derived metrics in the electorate profiles.

\section{Election results}\label{election-results}

The three type of vote count published for each federal election that
have been gathered and formatted for analysis in this study are as
follows:

\begin{itemize}
\item
  Division of preferences: distribution of preferences at each step of
  reallocation, beginning with first preferences.
\item
  Two party preferred: distribution of preferences where, by convention,
  comparisons are made between the ALP and the leading Liberal/National
  candidates. In seats where the final two candidates are not from the
  ALP and the Liberal or National parties, a two party preferred count
  may be conducted to find the result of preference flows to the ALP and
  the Liberal/National candidates. Vote swing percentage is calculated
  for this metric.
\item
  Two candidate preffered: distribution of preferences to the two
  candidates who came first and second in the election.
\end{itemize}

\section{Mapping Census Information to Electorates at Election
Times}\label{mapping-census-information-to-electorates-at-election-times}

In order to map Census information to each election, there are two cases
to consider: elections that fall on the same year as a Census, and those
that do not.

\subsubsection{Elections that fall on a Census
year}\label{elections-that-fall-on-a-census-year}

When a Census is conducted in an election year the CEDs used by the ABS
will match the AEC electoral divisions for that election, so the Census
profiles can be directly mapped to the electorates at election time.
This is done for 2001 and 2016.

\subsubsection{Elections that do not fall on a Census
year}\label{elections-that-do-not-fall-on-a-census-year}

If the election does not fall on the same year a Census is conducted,
data from the two nearest Censuses will be used to compute estimated
electorate profiles at election time. These will reflect both the
changes demographics over time, and the differences in electorate
boundaries between the election and either Census.

This presents a significant challenge, and requires the use of spatial
analytics and mapping tools to determine the composition of the
electorate. For each election and Census the AEC and ABS release a
shapefile (GIS map) of the electorate boundaries. Using tools
predominantly from the \(rgeos\) package, boundaries at election time
can be overlayed with those at Census time to determine what the
electorate boundary would have looked like, should it have been
implemented for the Census. This is done for each Census in order to
compute an estimated socio-demographic profile of this superimposed
boundary. With estimated profiles from each Census, the profile at
election time can be interpolated between the two Censuses, depending on
when the election falls.

(The algorithm to complete this projection is as follows - maybe this is
for a different section)

\appendix

\chapter{Additional stuff}\label{additional-stuff}

You might put some computer output here, or maybe additional tables.

Note that line 5 must appear before your first appendix. But other
appendices can just start like any other chapter.

\printbibliography[heading=bibintoc]



\end{document}
